\section{Uruchamianie i rozwój aplikacji}

Rozdział omawia proces uruchomienia i rozwoju - od pierwszego sklonowania projektu, przez uruchomienie w trybie deweloperskim, aż po budowanie wersji produkcyjnej.

\subsection{Wymagania systemowe}

Przed rozpoczęciem pracy z projektem należy zainstalować:

\begin{enumerate}
    \item \textbf{Node.js} (wersja 24.0.0 lub nowsza)
    \begin{itemize}
        \item Sprawdzenie wersji: \texttt{node -{}-version}
        \item Pobieranie: \url{https://nodejs.org}
    \end{itemize}

    \item \textbf{npm} (instalowany automatycznie z Node.js)
    \begin{itemize}
        \item Sprawdzenie wersji npm: \texttt{npm -{}-version}
    \end{itemize}

    \item \textbf{Git} (do klonowania repozytorium)
    \begin{itemize}
        \item Sprawdzenie wersji: \texttt{git -{}-version}
        \item Pobieranie: \url{https://git-scm.com}
    \end{itemize}
\end{enumerate}

\subsection{Pierwsze uruchomienie projektu}

\subsubsection{Krok 1: Instalacja zależności}

Po sklonowaniu lub pobraniu projektu, należy zainstalować wszystkie zależności:

\begin{lstlisting}[language=bash, caption={Instalacja zależności}]
npm install
\end{lstlisting}

Komenda ta:
\begin{itemize}
    \item Odczytuje plik \texttt{package.json}
    \item Pobiera wszystkie zależności z \href{https://www.npmjs.com/}{npm registry}
    \item Instaluje je w katalogu \texttt{node\_modules}
    \item Generuje lub aktualizuje lock file (\texttt{package-lock.json}) (który dokładnie opisuje strukturę i wersje wszystkich zainstalowanych zależności projektu)
\end{itemize}

\subsubsection{Krok 2: Konfiguracja zmiennych środowiskowych}

Skopiuj plik przykładowy \texttt{example.env} do \texttt{.env}:

\begin{lstlisting}[language=bash, caption={Utworzenie pliku .env}]
cp example.env .env
\end{lstlisting}

Lub ręcznie utwórz plik \texttt{.env} w katalogu głównym projektu:

\begin{lstlisting}[caption={Zawartość pliku .env}]
DATABASE_URL="file:./dev.db"
\end{lstlisting}

Dla SQLite wystarczy powyższa konfiguracja. Dla innych baz danych (PostgreSQL, MySQL) należy podać odpowiedni connection string.

\subsubsection{Krok 3: Inicjalizacja bazy danych}

Uruchom migracje Prisma aby utworzyć bazę danych i tabele:

\begin{lstlisting}[language=bash, caption={Wykonanie migracji}]
npm run db:update
\end{lstlisting}

Komenda ta wykonuje:
\begin{lstlisting}[language=bash]
prisma migrate deploy && prisma generate
\end{lstlisting}

\begin{itemize}
    \item \texttt{prisma migrate deploy} - wykonuje wszystkie oczekujące migracje, aktualizując strukturę bazy danych do najnowszej wersji. 
    \item \texttt{prisma generate} - generuje Prisma Client
\end{itemize}

Po wykonaniu tej komendy:
\begin{itemize}
    \item Zostanie utworzony plik \texttt{prisma/dev.db} (baza SQLite)
    \item Zostanie utworzona tabela \texttt{Todo}
    \item Prisma Client zostanie wygenerowany w \texttt{node\_modules/@prisma/client}
\end{itemize}

\subsubsection{Krok 4: Uruchomienie serwera deweloperskiego}

\begin{lstlisting}[language=bash, caption={Uruchomienie dev server}]
npm run dev
\end{lstlisting}

Komenda ta:
\begin{itemize}
    \item Uruchamia Next.js w trybie deweloperskim
    \item Startuje serwer na porcie 3000
    \item Włącza Hot Module Replacement (HMR)
    \item Włącza Fast Refresh (natychmiastowe odświeżanie po zmianach w kodzie)
\end{itemize}

Output w terminalu:
\begin{lstlisting}
> nextjs-todo@0.1.0 dev
> next dev

  ▲ Next.js 16.0.1
  - Local:        http://localhost:3000
  - Environments: .env

 ✓ Starting...
 ✓ Ready in 2.3s
\end{lstlisting}

Aplikacja jest dostępna pod adresem \url{http://localhost:3000}.

\subsection{Skrypty npm}

Wszystkie dostępne skrypty zdefiniowane w \texttt{package.json}:

\subsubsection{npm run dev - serwer deweloperski}

\begin{lstlisting}[language=bash]
npm run dev
\end{lstlisting}

\begin{itemize}
    \item Uruchamia Next.js w trybie development
    \item Port: 3000
    \item Hot Module Replacement włączony
    \item Source maps włączone
    \item Nie optymalizuje kodu (szybszy build)
\end{itemize}

\subsubsection{npm run build - budowanie produkcji}

\begin{lstlisting}[language=bash]
npm run build
\end{lstlisting}

\begin{itemize}
    \item Kompiluje aplikację do wersji produkcyjnej
    \item Optymalizuje kod (minifikacja, tree-shaking)
    \item Generuje statyczne HTML dla stron
    \item Tworzy skompresowane pliki JavaScript
\end{itemize}

\subsubsection{npm run start - serwer produkcyjny}

\begin{lstlisting}[language=bash]
npm run start
\end{lstlisting}

\begin{itemize}
    \item Uruchamia zbudowaną aplikację (wymaga wcześniejszego \texttt{npm run build})
    \item Port: 3000
    \item Optymalizowany kod
    \item Brak HMR
    \item Gotowe do wdrożenia
\end{itemize}

\subsubsection{npm run lint - analiza kodu}

\begin{lstlisting}[language=bash]
npm run lint
\end{lstlisting}

\begin{itemize}
    \item Uruchamia ESLint
    \item Sprawdza kod pod kątem błędów i niezgodności ze stylem
    \item Używa konfiguracji \texttt{eslint-config-next}
\end{itemize}

\subsubsection{npm run db:migrate - tworzenie migracji}

\begin{lstlisting}[language=bash]
npm run db:migrate nazwa_migracji
\end{lstlisting}

\begin{itemize}
    \item Tworzy nową migrację bazy danych
    \item Generuje plik SQL w \texttt{prisma/migrations}
    \item Nie wykonuje migracji (tylko tworzy plik)
\end{itemize}

Przykład:
\begin{lstlisting}[language=bash]
npm run db:migrate add_priority_field
\end{lstlisting}

\subsubsection{npm run db:update - wykonanie migracji}

\begin{lstlisting}[language=bash]
npm run db:update
\end{lstlisting}

\begin{itemize}
    \item Wykonuje wszystkie pending migrations
    \item Generuje Prisma Client
    \item Aktualizuje bazę danych do najnowszej wersji schematu
\end{itemize}

\subsection{Workflow deweloperski}

Typowy cykl pracy nad projektem:

\subsubsection{Uruchomienie środowiska}

\begin{lstlisting}[language=bash]
# Terminal 1: serwer deweloperski
npm run dev
\end{lstlisting}

Serwer uruchamia się i nasłuchuje na zmiany w plikach. Aby zakończyć użyj skrótu CTRL+C.

\subsubsection{Edycja kodu}

\begin{itemize}
    \item Otwórz plik w edytorze (np. \texttt{app/page.tsx})
    \item Wprowadź zmiany
    \item Zapisz plik
    \item Next.js automatycznie wykryje zmiany i odświeży przeglądarkę (Fast Refresh)
\end{itemize}

\subsubsection{Modyfikacja schematu bazy}

Jeśli zmieniasz schemat Prisma:

\begin{lstlisting}[language=bash]
# 1. Edytuj prisma/schema.prisma
# 2. Utwórz migrację
npm run db:migrate nazwa_zmiany

# 3. Wykonaj migrację
npm run db:update
\end{lstlisting}

\subsubsection{Sprawdzenie jakości kodu}

\begin{lstlisting}[language=bash]
npm run lint
\end{lstlisting}

Napraw ewentualne błędy zgłoszone przez ESLint.

\subsubsection{Testowanie}

\begin{itemize}
    \item Otwórz \url{http://localhost:3000} w przeglądarce
    \item Przetestuj funkcjonalności aplikacji
    \item Sprawdź konsolę przeglądarki (F12) pod kątem błędów
    \item Aby zakończyć użyj skrótu CTRL+C
\end{itemize}

\subsection{Prisma Studio - GUI bazy danych}

Prisma oferuje graficzny interfejs do przeglądania i edycji danych:

\begin{lstlisting}[language=bash]
npx prisma studio
\end{lstlisting}

Output:
\begin{lstlisting}
Environment variables loaded from .env
Prisma Studio is running on http://localhost:5555
\end{lstlisting}

W przeglądarce pod adresem \url{http://localhost:5555} dostępny jest graficzny interfejs do:
\begin{itemize}
    \item Przeglądania tabel
    \item Dodawania rekordów
    \item Edycji rekordów
    \item Usuwania rekordów
\end{itemize}

\subsection{Debugowanie}

\subsubsection{Console.log w Server Components}

W Server Components \texttt{console.log()} wypisuje do terminala (nie do konsoli przeglądarki):

\begin{lstlisting}[language=TypeScript, caption={Debugowanie Server Component}]
export default async function Home() {
  const todos = await getTodos();
  console.log('Liczba zadań:', todos.length);  // Output w terminalu
  // ...
}
\end{lstlisting}

\subsubsection{Console.log w Client Components}

W Client Components \texttt{console.log()} wypisuje do konsoli przeglądarki:

\begin{lstlisting}[language=TypeScript, caption={Debugowanie Client Component}]
'use client';

export function TodoItem({ id }) {
  const handleClick = () => {
    console.log('Kliknięto zadanie:', id);  // Output w konsoli przeglądarki
  };
  // ...
}
\end{lstlisting}

\subsubsection{React DevTools}

Zainstaluj rozszerzenie React DevTools do przeglądarki:
\begin{itemize}
    \item Chrome: \url{https://chrome.google.com/webstore}
    \item Firefox: \url{https://addons.mozilla.org}
\end{itemize}

Umożliwia:
\begin{itemize}
    \item Inspekcję drzewa komponentów
    \item Podgląd props i state
    \item Śledzenie renderowania komponentów
\end{itemize}

\subsection{Rozwiązywanie problemów}

\subsubsection{Błąd migracji}

Jeśli migracje są w niespójnym stanie:

\begin{lstlisting}[language=bash]
# Reset bazy (UWAGA: usuwa wszystkie dane!)
npx prisma migrate reset
\end{lstlisting}
\newpage
