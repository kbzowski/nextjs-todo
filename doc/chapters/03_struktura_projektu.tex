\section{Struktura projektu}

Niniejszy rozdział przedstawia organizację plików i katalogów w projekcie. Zrozumienie struktury projektu jest kluczowe dla efektywnej pracy z aplikacją Next.js.

\subsection{Drzewo katalogów}

Poniżej przedstawiono kompletną strukturę projektu (z pominięciem katalogów \texttt{node\_modules} oraz \texttt{.next}):

\begin{lstlisting}[caption={Struktura katalogów projektu}, basicstyle=\ttfamily\small, numbers=none]
nextjs-todo/
|
+-- app/                    # Katalog glowny aplikacji (App Router)
|   |
|   +-- components/         # Komponenty React wielokrotnego uzytku
|   |   +-- TodoForm.tsx
|   |   +-- TodoForm.module.css
|   |   +-- TodoItem.tsx
|   |   +-- TodoItem.module.css
|   |
|   +-- add/                # Strona dodawania zadania
|   |   +-- page.tsx
|   |   +-- add.module.css
|   |
|   +-- edit/               # Strona edycji zadania
|   |   +-- [id]/           # Dynamic route (parametr id)
|   |       +-- page.tsx
|   |       +-- edit.module.css
|   |
|   +-- actions.ts          # Server Actions (CRUD)
|   +-- layout.tsx          # Root Layout aplikacji
|   +-- page.tsx            # Strona glowna (/)
|   +-- page.module.css     # Style strony glownej
|   +-- globals.css         # Globalne style CSS
|   +-- favicon.ico         # Ikona aplikacji
|
+-- lib/                    # Pliki pomocnicze
|   +-- prisma.ts           # Singleton Prisma Client
|   +-- formatDate.ts       # Funkcja formatowania dat
|   +-- formData.ts         # Narzedzia do obslugi FormData
|
+-- prisma/                 # Konfiguracja bazy danych
|   +-- schema.prisma       # Schemat modeli danych
|   +-- dev.db              # Plik bazy SQLite (development)
|   +-- migrations/         # Historia migracji bazy
|
+-- public/                 # Zasoby statyczne
+-- package.json            # Zaleznosci i skrypty npm
+-- package-lock.json       # Lock file (npm)
+-- pnpm-lock.yaml          # Lock file (pnpm)
+-- tsconfig.json           # Konfiguracja TypeScript
+-- next.config.ts          # Konfiguracja Next.js
+-- .env                    # Zmienne srodowiskowe
+-- .gitignore              # Pliki ignorowane przez Git
+-- README.md               # Dokumentacja projektu
\end{lstlisting}

\subsection{Katalog app/}

Katalog \texttt{app/} to rdzeń aplikacji Next.js wykorzystującej App Router. Jego struktura bezpośrednio wpływa na routing aplikacji.

\subsubsection{Konwencje nazewnictwa}

Next.js wykorzystuje specjalne nazwy plików do określania ich roli:

\begin{itemize}
    \item \texttt{page.tsx} - definiuje stronę dostępną pod danym URL
    \item \texttt{layout.tsx} - definiuje layout współdzielony przez strony
    \item \texttt{loading.tsx} - komponent wyświetlany podczas ładowania (nieużywany w projekcie)
    \item \texttt{error.tsx} - komponent obsługi błędów (nieużywany w projekcie)
    \item \texttt{not-found.tsx} - strona 404 (nieużywana w projekcie)
\end{itemize}

\subsubsection{Routing oparty na plikach}

Struktura katalogów w \texttt{app/} definiuje routing:

\begin{table}[h]
\centering
\begin{tabular}{|l|l|}
\hline
\textbf{Ścieżka pliku} & \textbf{URL} \\
\hline
\texttt{app/page.tsx} & \texttt{/} \\
\texttt{app/add/page.tsx} & \texttt{/add} \\
\texttt{app/edit/[id]/page.tsx} & \texttt{/edit/1}, \texttt{/edit/2}, ... \\
\hline
\end{tabular}
\caption{Mapowanie struktury plików na URL}
\end{table}

\subsubsection{Dynamic Routes}

Katalog \texttt{app/edit/[id]/} reprezentuje \textit{dynamic route} - trasę z parametrem dynamicznym. Notacja \texttt{[id]} oznacza, że wartość tego segmentu URL jest przekazywana jako parametr do komponentu strony.

Przykład: URL \texttt{/edit/5} zostanie obsłużony przez \texttt{app/edit/[id]/page.tsx}, a parametr \texttt{id} otrzyma wartość \texttt{"5"}.

\subsection{Katalog app/components/}

Katalog \texttt{app/components/} zawiera komponenty React wielokrotnego użytku. Każdy komponent składa się z pliku TypeScript (\texttt{.tsx}) oraz opcjonalnie pliku styli CSS Modules (\texttt{.module.css}).

\subsubsection{TodoItem.tsx}

Komponent wyświetlający pojedyncze zadanie na liście. Jest to komponent kliencki (\texttt{'use client'}), ponieważ obsługuje interakcje użytkownika (zaznaczanie checkbox, klikanie przycisków).

Odwołanie do pliku: \texttt{app/components/TodoItem.tsx:1-48}

\subsubsection{TodoForm.tsx}

Komponent formularza używany zarówno do dodawania, jak i edycji zadań. Działa w dwóch trybach: \texttt{'add'} oraz \texttt{'edit'}. Również jest komponentem klienckim, ponieważ wykorzystuje hook \texttt{useActionState}.

Odwołanie do pliku: \texttt{app/components/TodoForm.tsx:1-61}

\subsection{Katalog lib/}

Katalog \texttt{lib/} zawiera pliki pomocnicze i narzędziowe wykorzystywane w różnych częściach aplikacji.

\subsubsection{lib/prisma.ts}

Plik eksportujący singleton instancji Prisma Client. Singleton pattern zapewnia, że w całej aplikacji istnieje tylko jedna instancja klienta bazy danych, co jest szczególnie istotne w środowisku deweloperskim (Hot Module Replacement w Next.js mógłby tworzyć wiele połączeń).

Odwołanie do pliku: \texttt{lib/prisma.ts}

\subsubsection{lib/formatDate.ts}

Plik zawierający funkcję \texttt{formatDate()}, która formatuje obiekt \texttt{Date} do czytelnego ciągu znaków w formacie polskim (DD/MM/YYYY HH:mm).

Odwołanie do pliku: \texttt{lib/formatDate.ts}

\subsubsection{lib/formData.ts}

Plik zawierający funkcje pomocnicze do pracy z obiektami \texttt{FormData}, np. ekstrakcję i walidację wartości z formularzy.

\subsection{Katalog prisma/}

Katalog \texttt{prisma/} zawiera wszystkie pliki związane z bazą danych i ORM Prisma.

\subsubsection{prisma/schema.prisma}

Plik definiujący schemat bazy danych - modele, relacje, konfigurację połączenia. Jest to deklaratywny opis struktury bazy danych.

Odwołanie do pliku: \texttt{prisma/schema.prisma:1-16}

\subsubsection{prisma/dev.db}

Plik bazy danych SQLite używany w środowisku deweloperskim. Zawiera wszystkie dane aplikacji.

\subsubsection{prisma/migrations/}

Katalog zawierający historię migracji bazy danych. Każda migracja jest zapisana w osobnym podkatalogu z plikiem SQL opisującym zmiany w schemacie.

\subsection{Pliki konfiguracyjne}

\subsubsection{package.json}

Plik definiujący metadane projektu, zależności (dependencies) oraz skrypty npm.

Odwołanie do pliku: \texttt{package.json:1-30}

\subsubsection{tsconfig.json}

Plik konfiguracyjny kompilatora TypeScript. Definiuje opcje kompilacji, ścieżki aliasów, poziom ścisłości typowania itp.

\subsubsection{next.config.ts}

Plik konfiguracyjny Next.js. W podstawowej wersji projektu jest pusty (używa domyślnej konfiguracji), ale może zawierać zaawansowane opcje konfiguracyjne frameworka.

\subsubsection{.env}

Plik zmiennych środowiskowych. Zawiera konfigurację wrażliwą (np. connection string do bazy danych), która nie powinna być commitowana do repozytorium Git.

Przykładowa zawartość:
\begin{lstlisting}[caption={Przykładowy plik .env}]
DATABASE_URL="file:./dev.db"
\end{lstlisting}

\subsection{Katalog public/}

Katalog \texttt{public/} zawiera zasoby statyczne dostępne publicznie (obrazy, fonty, ikony itp.). Pliki z tego katalogu są serwowane pod ścieżką główną domeny.
\newpage