\section{Stylowanie aplikacji}

Aplikacja wykorzystuje CSS Modules do stylowania komponentów. Niniejszy rozdział omawia system stylowania oraz konkretne style użyte w projekcie.

\subsection{CSS Modules}

CSS Modules to podejście do stylowania, które automatycznie izoluje style komponentów, zapobiegając konfliktom nazw klas CSS.

\subsubsection{Jak działają CSS Modules?}

\paragraph{1. Konwencja nazewnictwa}
Pliki CSS Modules mają rozszerzenie \texttt{.module.css}:
\begin{lstlisting}
page.module.css
TodoForm.module.css
TodoItem.module.css
\end{lstlisting}

\paragraph{2. Import i użycie}\mbox{}
\begin{lstlisting}[language=TypeScript, caption={Import CSS Modules}]
import styles from './page.module.css';

export default function Page() {
  return <div className={styles.container}>...</div>;
}
\end{lstlisting}

\paragraph{3. Generowanie unikalnych nazw}\mbox{}
Next.js automatycznie przekształca nazwy klas na unikalne:

\begin{lstlisting}[caption={Przed (w pliku CSS)}]
.container {
  max-width: 600px;
}
\end{lstlisting}

\begin{lstlisting}[caption={Po (w HTML)}]
<div class="page_container__a1b2c3">
\end{lstlisting}

Dzięki temu klasa \texttt{.container} w jednym module nie koliduje z \texttt{.container} w innym.

\subsubsection{Zalety CSS Modules}

\begin{itemize}
    \item \textbf{Scoped styles} - style ograniczone do komponentu
    \item \textbf{Brak konfliktów} - unikalne nazwy klas
    \item \textbf{Explicit dependencies} - komponenty jawnie importują swoje style
    \item \textbf{Dead code elimination} - nieużywane style można wykryć
    \item \textbf{Composition} - możliwość kompozycji klas z różnych modułów
\end{itemize}

\subsection{Globalne style - globals.css}

Plik \texttt{app/globals.css} zawiera globalne style stosowane do całej aplikacji. Importowany jest w root layout.

\begin{lstlisting}[language=CSS, caption={Fragmenty pliku globals.css}]
* {
  margin: 0;
  padding: 0;
  box-sizing: border-box;
}

body {
  font-family: -apple-system, BlinkMacSystemFont, 'Segoe UI',
    'Roboto', 'Oxygen', 'Ubuntu', 'Cantarell',
    'Fira Sans', 'Droid Sans', 'Helvetica Neue',
    sans-serif;
  background-color: #f5f5f5;
  color: #333;
  line-height: 1.6;
}

a {
  color: #0070f3;
  text-decoration: none;
}

a:hover {
  text-decoration: underline;
}
\end{lstlisting}

\subsubsection{Analiza globals.css}

\paragraph{CSS Reset}\mbox{}

\begin{lstlisting}[language=CSS]
* {
  margin: 0;
  padding: 0;
  box-sizing: border-box;
}
\end{lstlisting}

Resetuje domyślne marginesy i paddingi przeglądarki. \texttt{box-sizing: border-box} sprawia, że padding i border są wliczane w szerokość elementu.

\paragraph{Globalne style body}
\begin{itemize}
    \item \texttt{font-family} - system fonts (natywne fonty systemu operacyjnego)
    \item \texttt{background-color: \#f5f5f5} - jasne tło
    \item \texttt{color: \#333} - ciemnoszary tekst
    \item \texttt{line-height: 1.6} - większa czytelność tekstu
\end{itemize}

\paragraph{Style linków}\mbox{}
\begin{lstlisting}[language=CSS]
a {
  color: #0070f3;  /* niebieski */
  text-decoration: none;
}

a:hover {
  text-decoration: underline;
}
\end{lstlisting}

Linki są niebieskie bez podkreślenia. Po najechaniu myszką pojawia się podkreślenie.

\subsection{Style strony głównej - page.module.css}

Plik \texttt{app/page.module.css} zawiera style dla strony głównej.

\begin{lstlisting}[language=CSS, caption={Fragmenty page.module.css}]
.container {
  max-width: 600px;
  margin: 0 auto;
  padding: 20px;
}

.header {
  display: flex;
  justify-content: space-between;
  align-items: center;
  margin-bottom: 30px;
  padding-bottom: 20px;
  border-bottom: 2px solid #ddd;
}

.addBtn {
  background-color: #0070f3;
  color: white;
  padding: 10px 20px;
  border-radius: 5px;
  border: none;
  cursor: pointer;
  text-decoration: none;
}

.addBtn:hover {
  background-color: #0051cc;
}

.todoList {
  display: flex;
  flex-direction: column;
  gap: 10px;
}

.todoItem {
  background-color: white;
  padding: 15px;
  border-radius: 8px;
  box-shadow: 0 2px 4px rgba(0, 0, 0, 0.1);
  display: flex;
  justify-content: space-between;
  align-items: center;
}

.todoCheckbox {
  display: flex;
  align-items: center;
  gap: 10px;
  flex: 1;
  cursor: pointer;
}

.completed {
  text-decoration: line-through;
  color: #999;
}

.editBtn {
  background-color: #f39c12;
  color: white;
  padding: 5px 10px;
  border-radius: 4px;
  margin-right: 5px;
}

.deleteBtn {
  background-color: #e74c3c;
  color: white;
  padding: 5px 10px;
  border-radius: 4px;
  border: none;
  cursor: pointer;
}
\end{lstlisting}

\subsubsection{Analiza kluczowych klas}

\paragraph{.container - główny kontener}\mbox{}

\begin{lstlisting}[language=CSS]
.container {
  max-width: 600px;
  margin: 0 auto;
  padding: 20px;
}
\end{lstlisting}

\begin{itemize}
    \item \texttt{max-width: 600px} - ogranicza szerokość na dużych ekranach
    \item \texttt{margin: 0 auto} - wyśrodkowanie poziome
    \item \texttt{padding: 20px} - wewnętrzne odstępy
\end{itemize}

\paragraph{.header - nagłówek z przyciskiem}\mbox{}

\begin{lstlisting}[language=CSS]
.header {
  display: flex;
  justify-content: space-between;
  align-items: center;
}
\end{lstlisting}

Flexbox layout:
\begin{itemize}
    \item \texttt{space-between} - tytuł po lewej, przycisk po prawej
    \item \texttt{align-items: center} - pionowe wyśrodkowanie elementów
\end{itemize}

\paragraph{.todoList - lista zadań}\mbox{}

\begin{lstlisting}[language=CSS]
.todoList {
  display: flex;
  flex-direction: column;
  gap: 10px;
}
\end{lstlisting}

Flexbox column:
\begin{itemize}
    \item \texttt{flex-direction: column} - elementy jeden pod drugim
    \item \texttt{gap: 10px} - odstęp 10px między zadaniami
\end{itemize}

\paragraph{.todoItem - pojedyncze zadanie}\mbox{}

\begin{lstlisting}[language=CSS]
.todoItem {
  background-color: white;
  padding: 15px;
  border-radius: 8px;
  box-shadow: 0 2px 4px rgba(0, 0, 0, 0.1);
  display: flex;
  justify-content: space-between;
}
\end{lstlisting}

Styl karty (card):
\begin{itemize}
    \item Białe tło na szarym tle strony
    \item Zaokrąglone rogi (\texttt{border-radius})
    \item Subtelny cień (\texttt{box-shadow})
    \item Flexbox: checkbox po lewej, przyciski po prawej
\end{itemize}

\paragraph{.completed - ukończone zadanie}\mbox{}

\begin{lstlisting}[language=CSS]
.completed {
  text-decoration: line-through;
  color: #999;
}
\end{lstlisting}

Przekreślony tekst i jasnoszary kolor dla ukończonych zadań.

\paragraph{Przyciski - kolorystyka}
\begin{itemize}
    \item \texttt{.addBtn} - niebieski (\#0070f3)
    \item \texttt{.editBtn} - pomarańczowy (\#f39c12)
    \item \texttt{.deleteBtn} - czerwony (\#e74c3c)
\end{itemize}

Kolorystyka zgodna z konwencją:
\begin{itemize}
    \item Niebieski - akcja podstawowa (dodawanie)
    \item Pomarańczowy - akcja zmiany (edycja)
    \item Czerwony - akcja destrukcyjna (usuwanie)
\end{itemize}

\subsection{Style formularza - TodoForm.module.css}

Plik \texttt{app/components/TodoForm.module.css} styluje formularz dodawania/edycji.

\begin{lstlisting}[language=CSS, caption={Fragmenty TodoForm.module.css}]
.todoForm {
  background-color: white;
  padding: 30px;
  border-radius: 8px;
  box-shadow: 0 2px 8px rgba(0, 0, 0, 0.1);
}

.formGroup {
  margin-bottom: 20px;
}

.formGroup label {
  display: block;
  margin-bottom: 8px;
  font-weight: 600;
  color: #333;
}

.formGroup input {
  width: 100%;
  padding: 12px;
  border: 1px solid #ddd;
  border-radius: 4px;
  font-size: 16px;
}

.formGroup input:focus {
  outline: none;
  border-color: #0070f3;
  box-shadow: 0 0 0 3px rgba(0, 112, 243, 0.1);
}

.error {
  color: #dc2626;
  font-size: 14px;
  margin-top: 5px;
}

.errorMessage {
  background-color: #fef2f2;
  border: 1px solid #fecaca;
  color: #dc2626;
  padding: 12px;
  border-radius: 4px;
  margin-bottom: 20px;
}

.submitBtn {
  background-color: #0070f3;
  color: white;
  padding: 12px 24px;
  border: none;
  border-radius: 4px;
  cursor: pointer;
  font-size: 16px;
  font-weight: 600;
}

.submitBtn:hover {
  background-color: #0051cc;
}
\end{lstlisting}

\subsubsection{Analiza}

\paragraph{.formGroup - kontener pola}\mbox{}
Grupuje label i input razem z marginami.

\paragraph{input:focus - stan focus}\mbox{}

\begin{lstlisting}[language=CSS]
.formGroup input:focus {
  outline: none;
  border-color: #0070f3;
  box-shadow: 0 0 0 3px rgba(0, 112, 243, 0.1);
}
\end{lstlisting}

Po kliknięciu w pole:
\begin{itemize}
    \item Usuwa domyślny outline
    \item Zmienia kolor obramowania na niebieski
    \item Dodaje subtelną niebieską poświatę (box-shadow)
\end{itemize}

\paragraph{.error - błędy walidacji}\mbox{}

\begin{lstlisting}[language=CSS]
.error {
  color: #dc2626;  /* czerwony */
  font-size: 14px;
  margin-top: 5px;
}
\end{lstlisting}

Czerwone komunikaty błędów pod polami formularza.

\paragraph{.errorMessage - komunikat ogólny}\mbox{}

\begin{lstlisting}[language=CSS]
.errorMessage {
  background-color: #fef2f2;  /* jasny czerwony */
  border: 1px solid #fecaca;
  color: #dc2626;
  padding: 12px;
  border-radius: 4px;
}
\end{lstlisting}

Wyróżniona ramka z komunikatem błędu (np. błąd bazy danych).

\subsection{Responsive Design}

Aplikacja jest responsywna dzięki:

\begin{itemize}
    \item \textbf{Flexbox} - elastyczny layout dostosowujący się do rozmiaru ekranu
    \item \textbf{max-width} - ograniczenie szerokości na dużych ekranach
    \item \textbf{Relative units} - użycie \texttt{em}, \texttt{rem}, \texttt{\%} zamiast stałych pikseli
    \item \textbf{Mobile-first approach} - bazowe style działają na małych ekranach
    \item \textbf{Device Breakpoints} - aplikacja nie stosuje \href{https://shebang.pl/css/css-media-queries/}{Media Queries CSS}, ale ich wprowadzenie nie stanowi problemu.
\end{itemize}
\newpage