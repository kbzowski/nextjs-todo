\section{System routingu Next.js}

Routing to mechanizm mapowania URL na odpowiednie komponenty i strony aplikacji. Next.js wprowadza koncepcję \textit{file-system based routing} - struktury URL są definiowane przez strukturę plików i katalogów.

\subsection{App Router vs Pages Router}

Next.js oferuje dwa systemy routingu:

\begin{itemize}
    \item \textbf{Pages Router} - starszy system oparty na katalogu \texttt{pages/}
    \item \textbf{App Router} - nowszy system oparty na katalogu \texttt{app/} (używany w projekcie)
\end{itemize}

Nasza aplikacja wykorzystuje \textbf{App Router}, wprowadzony w Next.js 13, który oferuje zaawansowane funkcjonalności takie jak Server Components, nested layouts oraz streaming.

\subsection{Konwencje nazewnictwa}

App Router wykorzystuje specjalne nazwy plików do definiowania różnych typów zasobów:

\begin{table}[h]
\centering
\begin{tabular}{|l|l|}
\hline
\textbf{Nazwa pliku} & \textbf{Przeznaczenie} \\
\hline
\texttt{page.tsx} & Komponent strony dostępnej pod danym URL \\
\texttt{layout.tsx} & Layout współdzielony przez strony w danym segmencie \\
\texttt{loading.tsx} & Komponent wyświetlany podczas ładowania \\
\texttt{error.tsx} & Komponent obsługi błędów \\
\texttt{not-found.tsx} & Strona 404 (nie znaleziono) \\
\texttt{route.ts} & API Route Handler \\
\hline
\end{tabular}
\caption{Specjalne nazwy plików w App Router}
\end{table}

W naszym projekcie wykorzystujemy pliki \texttt{page.tsx} oraz \texttt{layout.tsx}.

\subsection{Mapowanie struktury na URL}

Struktura katalogów w \texttt{app/} bezpośrednio definiuje strukturę URL aplikacji.

\subsubsection{Przykłady mapowania}

\begin{table}[h]
\centering
\begin{tabular}{|l|l|l|}
\hline
\textbf{Plik} & \textbf{URL} & \textbf{Opis} \\
\hline
\texttt{app/page.tsx} & \texttt{/} & Strona główna \\
\texttt{app/add/page.tsx} & \texttt{/add} & Strona dodawania zadania \\
\texttt{app/edit/[id]/page.tsx} & \texttt{/edit/:id} & Strona edycji (dynamiczna) \\
\hline
\end{tabular}
\caption{Mapowanie plików na URL}
\end{table}

\subsection{Dynamic Routes - trasy dynamiczne}

Trasy dynamiczne umożliwiają tworzenie stron z parametrami w URL. Parametr dynamiczny definiuje się za pomocą nawiasów kwadratowych w nazwie katalogu.

\subsubsection{Przykład: app/edit/[id]/page.tsx}

Katalog \texttt{[id]} oznacza parametr dynamiczny. URL takie jak:
\begin{itemize}
    \item \texttt{/edit/1}
    \item \texttt{/edit/42}
    \item \texttt{/edit/xyz}
\end{itemize}

będą obsługiwane przez ten sam komponent, a wartość parametru będzie dostępna w komponencie.

\subsubsection{Dostęp do parametrów}

Parametry dynamiczne są przekazywane do komponentu strony jako props:

\begin{lstlisting}[language=TypeScript, caption={Dostęp do parametrów dynamicznych}]
export default async function EditPage({
  params,
}: {
  params: Promise<{ id: string }>;
}) {
  const { id } = await params;

  // Konwersja string na number
  const todoId = parseInt(id, 10);

  // Pobranie zadania z bazy
  const todo = await getTodoById(todoId);

  // ...
}
\end{lstlisting}

Odwołanie do pliku: \texttt{app/edit/[id]/page.tsx}

\paragraph{Uwaga o typach}
W Next.js 15+, parametry są zwracane jako \texttt{Promise}, co wymaga użycia \texttt{await} przed dostępem do wartości. Jest to związane z asynchronicznym renderowaniem.

\subsection{Layout - wspólny szablon}

Pliki \texttt{layout.tsx} definiują wspólny szablon HTML otaczający strony. Layout jest współdzielony przez wszystkie strony w danym segmencie i jego podsegmentach.

\subsubsection{Root Layout}

Plik \texttt{app/layout.tsx} to \textit{root layout} - główny szablon całej aplikacji. Musi zawierać tagi \texttt{<html>} oraz \texttt{<body>}.

\begin{lstlisting}[language=TypeScript, caption={Root layout aplikacji}]
import type { Metadata } from 'next';
import './globals.css';

export const metadata: Metadata = {
  title: 'TODO List',
  description: 'Aplikacja do zarządzania zadaniami',
};

export default function RootLayout({
  children,
}: Readonly<{
  children: React.ReactNode;
}>) {
  return (
    <html lang="pl">
      <body>{children}</body>
    </html>
  );
}
\end{lstlisting}

Odwołanie do pliku: \texttt{app/layout.tsx}

\subsubsection{Analiza root layout}

\begin{itemize}
    \item \texttt{metadata} - obiekt definiujący metadane strony (tytuł, opis, używane przez wyszukiwarki i karty social media)

    \item \texttt{lang="pl"} - atrybut języka dokumentu HTML (polski)

    \item \texttt{children} - prop zawierający zawartość renderowanej strony. Layout otacza wszystkie strony swoją strukturą.

    \item Import \texttt{globals.css} - globalne style CSS stosowane do całej aplikacji
\end{itemize}

\subsection{Nawigacja między stronami}

Next.js udostępnia komponent \texttt{Link} do nawigacji po aplikacji bez przeładowania strony (client-side navigation).

\subsubsection{Użycie komponentu Link}

\begin{lstlisting}[language=TypeScript, caption={Nawigacja z użyciem Link}]
import Link from 'next/link';

// Proste linki
<Link href="/">Strona główna</Link>
<Link href="/add">Dodaj zadanie</Link>

// Link z dynamicznym parametrem
<Link href={`/edit/${id}`}>Edytuj</Link>
\end{lstlisting}

Przykłady z projektu:
\begin{itemize}
    \item W \texttt{app/page.tsx}: link do \texttt{/add}
    \item W \texttt{app/components/TodoItem.tsx}: link do \texttt{/edit/\{id\}}
    \item W formularzach: link powrotu do \texttt{/}
\end{itemize}

Odwołania do plików:
\begin{itemize}
    \item \texttt{app/page.tsx:24-26} - przycisk "Dodaj nowe zadanie"
    \item \texttt{app/components/TodoItem.tsx:38-40} - przycisk "Edytuj"
\end{itemize}

\subsection{Nawigacja programowa}

Oprócz komponentu \texttt{Link}, Next.js oferuje funkcje do nawigacji programowej:

\subsubsection{redirect()}

Funkcja \texttt{redirect()} umożliwia przekierowanie użytkownika na inny URL. Używana w Server Actions po zakończeniu operacji.

\begin{lstlisting}[language=TypeScript, caption={Użycie redirect() w Server Action}]
import { redirect } from 'next/navigation';

export async function saveTodo(prevState: FormState, formData: FormData) {
  // ... walidacja i zapis do bazy ...

  revalidatePath('/');
  redirect('/');  // Przekierowanie na stronę główną
}
\end{lstlisting}

Odwołanie do pliku: \texttt{app/actions.ts:72-73}

\subsubsection{useRouter() (client-side)}

W komponentach klienckich można użyć hooka \texttt{useRouter()}:

\begin{lstlisting}[language=TypeScript, caption={Client-side routing z useRouter}]
'use client';
import { useRouter } from 'next/navigation';

function MyComponent() {
  const router = useRouter();

  const handleClick = () => {
    router.push('/add');
  };

  return <button onClick={handleClick}>Dodaj</button>;
}
\end{lstlisting}

\subsection{Cache Revalidation}

Next.js automatycznie cache'uje renderowane strony i dane. Funkcja \texttt{revalidatePath()} pozwala unieważnić cache dla danej ścieżki.

\begin{lstlisting}[language=TypeScript, caption={Revalidacja cache}]
import { revalidatePath } from 'next/cache';

export async function toggleTodo(id: number) {
  // ... operacja na bazie danych ...

  revalidatePath('/');  // Unieważnia cache strony głównej
}
\end{lstlisting}

Odwołanie do pliku: \texttt{app/actions.ts:92}

Po wywołaniu \texttt{revalidatePath('/')}, przy następnym odwiedzeniu strony głównej Next.js ponownie wyrenderuje ją po stronie serwera z aktualnymi danymi z bazy.

\subsection{Struktura routingu w projekcie}

Aplikacja TODO posiada trzy główne trasy:

\begin{enumerate}
    \item \textbf{Strona główna (/)}: wyświetla listę wszystkich zadań
    \begin{itemize}
        \item Plik: \texttt{app/page.tsx}
        \item Typ: Server Component
        \item Pobiera dane z bazy przy każdym renderowaniu
    \end{itemize}

    \item \textbf{Dodawanie zadania (/add)}: formularz dodawania nowego zadania
    \begin{itemize}
        \item Plik: \texttt{app/add/page.tsx}
        \item Typ: Server Component (kontener dla Client Component)
        \item Zawiera komponent TodoForm w trybie 'add'
    \end{itemize}

    \item \textbf{Edycja zadania (/edit/[id])}: formularz edycji istniejącego zadania
    \begin{itemize}
        \item Plik: \texttt{app/edit/[id]/page.tsx}
        \item Typ: Server Component z dynamic route
        \item Pobiera dane zadania z bazy na podstawie parametru id
        \item Zawiera komponent TodoForm w trybie 'edit'
    \end{itemize}
\end{enumerate}
\newpage