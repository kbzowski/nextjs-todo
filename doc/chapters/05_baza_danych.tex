\section{Baza danych i Prisma ORM}

Niniejszy rozdział omawia warstwę persystencji danych w aplikacji. Szczegółowo opisuje schemat bazy danych, model danych oraz mechanizm migracji.

\subsection{Wprowadzenie do Prisma}

Prisma to nowoczesny ORM (Object-Relational Mapping), który zapewnia warstwę abstrakcji nad bazą danych. Zamiast pisać surowe zapytania SQL, używamy API Prisma w TypeScript, które jest w pełni typowane.

\subsubsection{Zalety Prisma}

\begin{enumerate}
    \item \textbf{Type Safety} - zapytania są w pełni typowane, kompilator wykryje błędy
    \item \textbf{Auto-completion} - IDE podpowiada dostępne pola i metody
    \item \textbf{Migracje} - automatyczne generowanie i zarządzanie migracjami schematu. Migracja to uporządkowana zmiana w strukturze bazy danych (np. dodanie tabeli, kolumny, indeksu). Prisma automatycznie generuje pliki SQL opisujące te zmiany, co pozwala na kontrolowanie ewolucji schematu bazy danych w czasie oraz synchronizację między środowiskami (development, staging, production)
    \item \textbf{Deklaratywny schemat} - czytelna definicja modeli w języku Prisma Schema
\end{enumerate}

\subsection{Schemat Prisma}

Schemat Prisma definiuje strukturę bazy danych w pliku \texttt{prisma/schema.prisma}. Jest to deklaratywna definicja modeli, relacji oraz konfiguracji połączenia.

\begin{lstlisting}[caption={Pełna zawartość pliku schema.prisma}, label={lst:prisma-schema}]
generator client {
  provider = "prisma-client-js"
}

datasource db {
  provider = "sqlite"
  url      = env("DATABASE_URL")
}

model Todo {
  id        Int      @id @default(autoincrement())
  text      String
  completed Boolean  @default(false)
  createdAt DateTime @default(now())
}
\end{lstlisting}

Odwołanie do pliku: \texttt{prisma/schema.prisma:1-16}

\subsubsection{Sekcja generator}

\begin{lstlisting}[language=JavaScript]
generator client {
  provider = "prisma-client-js"
}
\end{lstlisting}

Definicja generatora klienta Prisma. Opcja \texttt{provider = "prisma-client-js"} oznacza, że zostanie wygenerowany klient JavaScript/TypeScript.

Po każdej zmianie schematu należy uruchomić:
\begin{lstlisting}[language=bash]
npx prisma generate
\end{lstlisting}

Komenda ta generuje kod TypeScript klienta Prisma w katalogu \texttt{node\_modules/@prisma/client}.

\subsubsection{Sekcja datasource}

\begin{lstlisting}[language=JavaScript]
datasource db {
  provider = "sqlite"
  url      = env("DATABASE_URL")
}
\end{lstlisting}

Definicja źródła danych (bazy danych):

\begin{itemize}
    \item \texttt{provider = "sqlite"} - typ bazy danych (SQLite)
    \item \texttt{url = env("DATABASE\_URL")} - connection string pobierany ze zmiennej środowiskowej
\end{itemize}

Zmienna \texttt{DATABASE\_URL} jest zdefiniowana w pliku \texttt{.env}:
\begin{lstlisting}
DATABASE_URL="file:./dev.db"
\end{lstlisting}

\subsection{Model Todo}

Model Todo reprezentuje pojedyncze zadanie na liście. Składa się z czterech pól:

\begin{lstlisting}[caption={Model Todo}, label={lst:model-todo}]
model Todo {
  id        Int      @id @default(autoincrement())
  text      String
  completed Boolean  @default(false)
  createdAt DateTime @default(now())
}
\end{lstlisting}

Odwołanie do pliku: \texttt{prisma/schema.prisma:10-15}

\subsubsection{Pola modelu}

\paragraph{id: Int}
\begin{itemize}
    \item Typ: Integer (liczba całkowita)
    \item \texttt{@id} - oznacza klucz główny tabeli
    \item \texttt{@default(autoincrement())} - automatycznie inkrementowana wartość dla nowych rekordów
    \item Unikalna wartość identyfikująca zadanie
\end{itemize}

\paragraph{text: String}
\begin{itemize}
    \item Typ: String (tekst)
    \item Przechowuje treść zadania
    \item Brak domyślnej wartości - pole wymagane przy tworzeniu rekordu
\end{itemize}

\paragraph{completed: Boolean}
\begin{itemize}
    \item Typ: Boolean (wartość logiczna)
    \item \texttt{@default(false)} - domyślnie zadanie jest nieukończone
    \item Określa, czy zadanie zostało wykonane
\end{itemize}

\paragraph{createdAt: DateTime}
\begin{itemize}
    \item Typ: DateTime (data i czas)
    \item \texttt{@default(now())} - automatycznie ustawiana na bieżący czas przy tworzeniu rekordu
    \item Przechowuje datę i czas utworzenia zadania
\end{itemize}

\subsection{Migracje bazy danych}

Migracje to mechanizm wersjonowania schematu bazy danych. Każda zmiana w schemacie Prisma generuje migrację - plik SQL opisujący transformację schematu.

\subsubsection{Inicjalna migracja}

Pierwsza migracja projektu znajduje się w katalogu:
\begin{lstlisting}
prisma/migrations/20251109083508_init/migration.sql
\end{lstlisting}

Zawartość pliku migracji:

\begin{lstlisting}[language=SQL, caption={Inicjalna migracja bazy danych}]
-- CreateTable
CREATE TABLE "Todo" (
    "id" INTEGER NOT NULL PRIMARY KEY AUTOINCREMENT,
    "text" TEXT NOT NULL,
    "completed" BOOLEAN NOT NULL DEFAULT false,
    "createdAt" DATETIME NOT NULL DEFAULT CURRENT_TIMESTAMP
);
\end{lstlisting}

\subsubsection{Analiza SQL}

\begin{itemize}
    \item \texttt{CREATE TABLE "Todo"} - tworzy tabelę o nazwie "Todo"
    \item \texttt{INTEGER NOT NULL PRIMARY KEY AUTOINCREMENT} - klucz główny typu całkowitego, automatycznie inkrementowany
    \item \texttt{TEXT NOT NULL} - pole tekstowe, wymagane (nie może być NULL)
    \item \texttt{BOOLEAN NOT NULL DEFAULT false} - pole logiczne z wartością domyślną false
    \item \texttt{DATETIME NOT NULL DEFAULT CURRENT\_TIMESTAMP} - data/czas z automatyczną wartością bieżącego czasu
\end{itemize}

\subsubsection{Proces migracji}

Typowy workflow pracy z migracjami:

\begin{enumerate}
    \item Modyfikacja pliku \texttt{schema.prisma} (np. dodanie nowego pola)
    \item Utworzenie migracji: \texttt{npx prisma migrate dev -{}-name nazwa\_migracji}
    \item Prisma automatycznie:
    \begin{itemize}
        \item Generuje plik SQL z migracją
        \item Wykonuje migrację na bazie deweloperskiej
        \item Regeneruje Prisma Client
    \end{itemize}
\end{enumerate}

\subsection{Prisma Client}

Prisma Client to automatycznie generowany klient bazy danych, który zapewnia type-safe API do wykonywania zapytań.

\subsubsection{Singleton Prisma Client}

Plik \texttt{lib/prisma.ts} eksportuje singleton instancji Prisma Client:

\begin{lstlisting}[language=TypeScript, caption={Singleton Prisma Client}]
import { PrismaClient } from '@prisma/client';

const globalForPrisma = globalThis as unknown as {
  prisma: PrismaClient | undefined;
};

export const prisma = globalForPrisma.prisma ?? new PrismaClient();

if (process.env.NODE_ENV !== 'production') {
  globalForPrisma.prisma = prisma;
}
\end{lstlisting}

\subsubsection{Dlaczego singleton?}

W środowisku deweloperskim Next.js używa Hot Module Replacement (HMR), który przeładowuje moduły podczas developmentu. Bez singletona każde przeładowanie tworzyłoby nową instancję \texttt{PrismaClient}, co prowadziłoby do wyczerpania połączeń z bazą danych.

Singleton zapewnia, że w całej aplikacji istnieje tylko jedna instancja klienta.

\subsection{Przykłady użycia Prisma Client}

\subsubsection{Pobieranie wszystkich zadań}

\begin{lstlisting}[language=TypeScript, caption={Pobieranie wszystkich zadań}]
const todos = await prisma.todo.findMany({
  orderBy: {
    createdAt: 'desc',
  },
});
\end{lstlisting}

\subsubsection{Tworzenie nowego zadania}

\begin{lstlisting}[language=TypeScript, caption={Tworzenie zadania}]
await prisma.todo.create({
  data: {
    text: "Kupić mleko"
  },
});
\end{lstlisting}

\subsubsection{Aktualizacja zadania}

\begin{lstlisting}[language=TypeScript, caption={Aktualizacja zadania}]
await prisma.todo.update({
  where: { id: 1 },
  data: { completed: true },
});
\end{lstlisting}

\subsubsection{Usunięcie zadania}

\begin{lstlisting}[language=TypeScript, caption={Usunięcie zadania}]
await prisma.todo.delete({
  where: { id: 1 },
});
\end{lstlisting}

\subsection{Typy generowane przez Prisma}

Prisma automatycznie generuje typy TypeScript dla modeli:

\begin{lstlisting}[language=TypeScript, caption={Automatycznie generowane typy}]
import type { Todo } from '@prisma/client';

// Typ Todo jest automatycznie wygenerowany:
// {
//   id: number;
//   text: string;
//   completed: boolean;
//   createdAt: Date;
// }
\end{lstlisting}

Typy te zapewniają bezpieczeństwo typów w całej aplikacji.
\newpage