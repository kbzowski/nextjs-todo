\section{Wprowadzenie}

\subsection{Cel dokumentu}

Niniejszy dokument ma na celu przedstawienie procesu tworzenia nowoczesnej aplikacji webowej od podstaw. Materiał przeznaczony jest dla studentów informatyki, którzy rozpoczynają swoją przygodę z programowaniem aplikacji internetowych. Dokument szczegółowo omawia wszystkie aspekty budowy aplikacji do zarządzania listą zadań (ang. \textit{TODO list}), od konfiguracji projektu, przez implementację logiki biznesowej, aż po stylowanie interfejsu użytkownika.

\subsection{Czym jest aplikacja webowa?}

Aplikacja webowa to program komputerowy, który działa w przeglądarce internetowej i umożliwia użytkownikowi interakcję poprzez interfejs graficzny dostępny za pośrednictwem protokołu HTTP/HTTPS. W przeciwieństwie do tradycyjnych aplikacji desktopowych, aplikacje webowe nie wymagają instalacji na komputerze użytkownika - wystarczy przeglądarka internetowa.

Współczesne aplikacje webowe można podzielić na kilka kategorii:

\begin{itemize}
    \item \textbf{Aplikacje Single Page Application (SPA)} - aplikacje działające w ramach jednej strony HTML, gdzie zawartość jest dynamicznie aktualizowana bez przeładowania całej strony. Przykłady: Gmail, Facebook, Twitter.

    \item \textbf{Aplikacje z renderowaniem po stronie serwera (SSR - Server-Side Rendering)} - aplikacje, w których HTML jest generowany na serwerze i wysyłany do przeglądarki. Zapewnia to lepsze SEO oraz szybszy pierwszy rendering strony.

    \item \textbf{Aplikacje hybrydowe} - łączą zalety SPA i SSR, umożliwiając renderowanie niektórych stron na serwerze, a innych w przeglądarce. Taki model został zastosowany w prezentowanej aplikacji.
\end{itemize}

\subsection{Architektura full-stack}

Aplikacja full-stack składa się z dwóch głównych warstw:

\begin{enumerate}
    \item \textbf{Frontend (warstwa prezentacji)} - część aplikacji działająca w przeglądarce użytkownika, odpowiedzialna za wyświetlanie interfejsu graficznego i obsługę interakcji użytkownika. W prezentowanej aplikacji frontend został zbudowany przy użyciu biblioteki React oraz frameworka Next.js.

    \item \textbf{Backend (warstwa serwera)} - część aplikacji działająca na serwerze, odpowiedzialna za przetwarzanie logiki biznesowej, dostęp do bazy danych oraz obsługę żądań HTTP. W naszej aplikacji backend został zrealizowany przy użyciu mechanizmu \textit{Server Actions} frameworka Next.js.
\end{enumerate}

\subsection{Omówienie aplikacji}

Prezentowana aplikacja TODO to pełnofunkcjonalna aplikacja webowa umożliwiająca zarządzanie listą zadań. Użytkownik może:

\begin{itemize}
    \item Dodawać nowe zadania
    \item Edytować istniejące zadania
    \item Oznaczać zadania jako ukończone
    \item Usuwać zadania z listy
    \item Przeglądać listę wszystkich zadań posortowanych według daty utworzenia
\end{itemize}

Aplikacja została zbudowana w oparciu o nowoczesne technologie webowe:

\begin{itemize}
    \item \textbf{Next.js 16} - framework React do budowy aplikacji webowych
    \item \textbf{React 19} - biblioteka JavaScript do budowy interfejsów użytkownika
    \item \textbf{TypeScript 5} - typowany nadzbiór JavaScript zapewniający bezpieczeństwo typów
    \item \textbf{Prisma ORM} - narzędzie do zarządzania bazą danych
    \item \textbf{SQLite} - lekka baza danych SQL
    \item \textbf{CSS Modules} - system stylowania zapewniający izolację styli
\end{itemize}

\subsection{Wymagania wstępne}

Aby móc samodzielnie zbudować i uruchomić prezentowaną aplikację, student powinien posiadać:

\begin{enumerate}
    \item Podstawową znajomość języka JavaScript
    \item Podstawową znajomość HTML i CSS
    \item Zainstalowane środowisko Node.js (w wersji 18 lub nowszej)
    \item Zainstalowany menedżer pakietów npm lub pnpm
    \item Edytor kodu (np. Visual Studio Code, WebStorm)
    \item Przeglądarkę internetową (np. Chrome, Firefox, Edge)
\end{enumerate}

\subsection{Struktura dokumentu}

Dokument został podzielony na rozdziały omawiające poszczególne aspekty aplikacji:

\begin{itemize}
    \item Rozdział 2 omawia wykorzystane technologie i ich rolę w aplikacji
    \item Rozdział 3 przedstawia strukturę projektu i organizację plików
    \item Rozdział 4 opisuje pliki konfiguracyjne projektu
    \item Rozdział 5 omawia model danych i integrację z bazą danych
    \item Rozdział 6 wyjaśnia system routingu w Next.js
    \item Rozdziały 7-8 szczegółowo opisują komponenty React
    \item Rozdział 9 omawia implementację logiki biznesowej
    \item Rozdział 10 przedstawia stylowanie aplikacji
    \item Rozdział 11 wyjaśnia proces uruchamiania i developmentu
\end{itemize}
\newpage